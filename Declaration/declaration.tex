% ******************************* Thesis Declaration ********************************

\begin{declaration}

Dieses Skript wurde erstellt, da eine vernünftige Zusammenfassung zur Informatik I Vorlesung an der Hochschule Karlsruhe zum Zeitpunkt dieser Ausarbeitung nicht vorlag. Das Ziel dieses Skripts ist es, eine grobe Zusammenfassung der am Anfang besprochenen Inhalte zu geben und Studenten das Lernen auf die Prüfung zu vereinfachen. 

Das Skript besteht aus drei Teilen. Der erste Teil beschäftigt sich mit den grundlegenden Datentypen, mit der internen Verwaltung von gewissen Datentypen bei Java und der Umrechnung verschiedener Zahlensysteme. Dieser Teil wurde anhand der in der Vorlesung besprochenen Verfahren zusammengeschrieben.

Der zweite Teil ist eine Zusammenfassung der Programmierkonventionen in der Programmiersprache Java. Der Inhalt ist von der Webseite \url{http://www.home.hs-karlsruhe.de/~pach0003/informatik_1/java_richtlinien/einleitung.html#uebersicht} genommen, welche von Prof. Dr. Pape angefertigt wurde. 

Der dritte Teil behandelt die Speicherverwaltung mittels Heap und Stack und die Modellierungsmöglichkeiten anhand der von UML zur Verfügung gestellten Aktivitäts- und Klassendiagramme. Hierzu wurden primär andere Quellen als Basis genommen, da die in der Vorlesung vorgestellten Konzepte sehr kurz gefasst wurden.

Dieses Skript ist ausschließlich zur privaten Nutzung gedacht, die Korrektheit der hier aufgeführten Informationen kann nicht gewährleistet werden. Solltest Du Fehler finden, werde ich diese gerne korrigieren.
Ich hoffe, dass Dir dieses Skript beim Lernen für die Klausur hilft.
% Author and date will be inserted automatically from thesis.tex \author \degreedate

\end{declaration}

